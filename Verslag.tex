\documentclass[11pt]{report}
\usepackage[a4paper, top=3cm, bottom=3cm]{geometry}

\usepackage[dutch]{babel}
\selectlanguage{dutch}

\usepackage{color}
\definecolor{gray75}{gray}{0.75}

\usepackage{framed}

\usepackage{graphicx}
\usepackage{tikz}

\usepackage{amsmath}

\usepackage[colorlinks=true,linkcolor=black]{hyperref}

\usepackage[T1]{fontenc}
\usepackage{titlesec}

\newcommand{\hsp}{\hspace{15pt}}

\titleformat{\chapter}[hang]{\Huge\bfseries}{\thechapter\hsp\textcolor{gray75}{|}\hsp}{0pt}{\Huge\bfseries}
\titlespacing{\chapter}{0pt}{-3.5ex plus 1ex minus .2ex}{2.3em plus .2ex}

\newlength{\leftbarwidth}
\setlength{\leftbarwidth}{3pt}
\newlength{\leftbarsep}
\setlength{\leftbarsep}{10pt}

\newcommand*{\leftbarcolorcmd}{\color{leftbarcolor}}% as a command to be more flexible
\colorlet{leftbarcolor}{black}

\renewenvironment{leftbar}{%
    \def\FrameCommand{{\leftbarcolorcmd{\vrule width \leftbarwidth\relax\hspace {\leftbarsep}}}}%
    \MakeFramed {\advance \hsize -\width \FrameRestore }}{\endMakeFramed}

\begin{document}

\begin{titlepage}
\begin{center}
{\huge\bfseries A New Field-Effect Transistor with Selectively Doped $GaAs/n-Al_{x}Ga_{1-x}As$ Heterojunction}
\\ \bigskip
{\Large Discussie \& kwalitatieve analyse}
\end{center}
\vfill
\begin{flushleft}
\setlength{\leftbarwidth}{1pt}
\colorlet{leftbarcolor}{gray75}
\begin{leftbar}
Halfgeleiders \\
Jelle Verstraaten, 500236946 \\
\texttt{jelle@benext.nl} \\
Erik Steuten\\
E-technology \\
{\small \today} \\
\end{leftbar}
\end{flushleft}
\end{titlepage}

\tableofcontents
%\listoffigures
%\listoftables

\chapter{Inleiding}

\chapter{Discussie}

\chapter{Resultaten}

\chapter{Referenties}

\bibliographystyle{IEEEtran}
\bibliography{literatuur/literature}

\appendix

\end{document}

%Verslag
%- Over het artikel
%- Minimaal 1 referentie uit het artikel -> Wat moeten we er mee?
%- Goed lopend verhaal
%- Hoe lang moet het verslag zijn?
%- Voeg een kopie van het artikel toe
%- Voeg een kopie toe van het artikel waarnaar het meeste gerefereert wordt.
%- Analoge EN digitale kopie
%
%- Inleiding:
%  - Wie heeft het geschreven, waarom, wanneer
%  - Zijn er meer referenties, vergelijkbare onderzoeken?
%- Discussie:
%  - Zijn de onderzoeksvragen beantwoord?
%  - Hoe goed waren de methodes die in het onderzoek zijn gebruikt?
%  - Wat is de betekenis van de resultaten en waarom?
%  - Hoe passen de resultaten in wat er al bekend is?
%
%Beantwoord de volgende vragen:
%- Inleiding:
%  - Kijk of het artikel in google scholar staat
%  - Waar werkt de auteur? -> Wat is dat voor instituut?
%  - Door wie wordt het onderzoek gefinancieerd?
%  - Is het artikel gepeer-reviewd?
%  - Zijn er meerdere bronnen die tot hetzelfde resultaat komen?
%  - Wordt er verwezen naar eerdere onderzoeken op hetzelfde gebied?
%  - Is het abstract duidelijk en nuttig voor jou?
%  - Kun je met het lezen van de inleiding de context van het artikel achterhalen?
%- Discussie:
%  - Zijn de onderzoeksvragen beantwoord?
%  - Hoe goed waren de methodes die in het onderzoek zijn gebruikt?
%  - Wat is de betekenis van de resultaten en waarom?
%  - Hoe passen de resultaten in wat er al bekend is?
%  - Hoe dragen de resultaten bij aan het orginele doel van het onderzoek?
%- Resultaten:
%  - Gedetaileerde beschrijving van de resultatenin tabellen, figuren, etc.
%  - Laat zien waar je dit terugvindt in het artikel
%- Experimenten:
%  - Welke opstellingen zijn er gebruikt?
%  - Welke methoden zijn er gebruikt?
%- Referenties:
%  - Welke methode wordt gebruikt om referenties weer te geven?
%  - Zoek de referentie waarnaar het artikel het meeste refereert, voeg een kopie toe
%
%
%Inleverdatum: 2015-06-22
%Herkansing: 2015-08-31
